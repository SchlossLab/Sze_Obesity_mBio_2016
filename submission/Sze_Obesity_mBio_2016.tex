\documentclass[12pt,]{article}
\usepackage{lmodern}
\usepackage{amssymb,amsmath}
\usepackage{ifxetex,ifluatex}
\usepackage{fixltx2e} % provides \textsubscript
\ifnum 0\ifxetex 1\fi\ifluatex 1\fi=0 % if pdftex
  \usepackage[T1]{fontenc}
  \usepackage[utf8]{inputenc}
\else % if luatex or xelatex
  \ifxetex
    \usepackage{mathspec}
    \usepackage{xltxtra,xunicode}
  \else
    \usepackage{fontspec}
  \fi
  \defaultfontfeatures{Mapping=tex-text,Scale=MatchLowercase}
  \newcommand{\euro}{€}
\fi
% use upquote if available, for straight quotes in verbatim environments
\IfFileExists{upquote.sty}{\usepackage{upquote}}{}
% use microtype if available
\IfFileExists{microtype.sty}{%
\usepackage{microtype}
\UseMicrotypeSet[protrusion]{basicmath} % disable protrusion for tt fonts
}{}
\usepackage[margin=1.0in]{geometry}
\ifxetex
  \usepackage[setpagesize=false, % page size defined by xetex
              unicode=false, % unicode breaks when used with xetex
              xetex]{hyperref}
\else
  \usepackage[unicode=true]{hyperref}
\fi
\hypersetup{breaklinks=true,
            bookmarks=true,
            pdfauthor={},
            pdftitle={Looking for a Signal in the Noise: Revisiting Obesity and the Microbiome},
            colorlinks=true,
            citecolor=blue,
            urlcolor=blue,
            linkcolor=magenta,
            pdfborder={0 0 0}}
\urlstyle{same}  % don't use monospace font for urls
\usepackage{graphicx,grffile}
\makeatletter
\def\maxwidth{\ifdim\Gin@nat@width>\linewidth\linewidth\else\Gin@nat@width\fi}
\def\maxheight{\ifdim\Gin@nat@height>\textheight\textheight\else\Gin@nat@height\fi}
\makeatother
% Scale images if necessary, so that they will not overflow the page
% margins by default, and it is still possible to overwrite the defaults
% using explicit options in \includegraphics[width, height, ...]{}
\setkeys{Gin}{width=\maxwidth,height=\maxheight,keepaspectratio}
\setlength{\parindent}{0pt}
\setlength{\parskip}{6pt plus 2pt minus 1pt}
\setlength{\emergencystretch}{3em}  % prevent overfull lines
\providecommand{\tightlist}{%
  \setlength{\itemsep}{0pt}\setlength{\parskip}{0pt}}
\setcounter{secnumdepth}{0}

%%% Use protect on footnotes to avoid problems with footnotes in titles
\let\rmarkdownfootnote\footnote%
\def\footnote{\protect\rmarkdownfootnote}

%%% Change title format to be more compact
\usepackage{titling}

% Create subtitle command for use in maketitle
\newcommand{\subtitle}[1]{
  \posttitle{
    \begin{center}\large#1\end{center}
    }
}

\setlength{\droptitle}{-2em}
  \title{Looking for a Signal in the Noise: Revisiting Obesity and the Microbiome}
  \pretitle{\vspace{\droptitle}\centering\huge}
  \posttitle{\par}
  \author{}
  \preauthor{}\postauthor{}
  \date{}
  \predate{}\postdate{}

\usepackage{helvet} % Helvetica font
\renewcommand*\familydefault{\sfdefault} % Use the sans serif version of the font
\usepackage[T1]{fontenc}

\usepackage[none]{hyphenat}

\usepackage{setspace}
\doublespacing
\setlength{\parskip}{1em}

\usepackage{lineno}

\usepackage{pdfpages}
\usepackage{comment}

% Redefines (sub)paragraphs to behave more like sections
\ifx\paragraph\undefined\else
\let\oldparagraph\paragraph
\renewcommand{\paragraph}[1]{\oldparagraph{#1}\mbox{}}
\fi
\ifx\subparagraph\undefined\else
\let\oldsubparagraph\subparagraph
\renewcommand{\subparagraph}[1]{\oldsubparagraph{#1}\mbox{}}
\fi

\begin{document}
\maketitle

\section{Lack of power and small effect size confounds the ability to
differentiate non-obese and obese individuals using gut microbiome
data}\label{lack-of-power-and-small-effect-size-confounds-the-ability-to-differentiate-non-obese-and-obese-individuals-using-gut-microbiome-data}

\begin{center}
\vspace{25mm}


Running Title: The Human Microbiome and Obesity

\vspace{10mm}

Marc A Sze and Patrick D Schloss${^\dagger}$

\vspace{10mm}

Contributions: Both authors contributed to the planning, design, execution, interpretation, and writing of the analyses.


\vspace{20mm}

$\dagger$ To whom correspondence should be addressed: pschloss@umich.edu

Department of Microbiology and Immunology, University of Michigan, Ann Arbor, MI
\end{center}

\newpage

\linenumbers

\subsection{Abstract}\label{abstract}

Two recent studies have re-analyzed published data and found that when
datasets are analyzed independently there was limited support the widely
accepted hypothesis that changes in the microbiome are associated with
obesity. This hypothesis was reconsidered by increasing the number of
data sets and pooling the results across the individual datasets. The
Preferred Reporting Items for Systematic Reviews and Meta-Analyses
(PRISMA) guidelines were applied to identify 10 studies for an updated
and more synthetic analysis. Alpha diversity metrics and the relative
risk of obesity based on those metrics were used to identify a limited
number of significant associations with obesity; however, when the
results of the studies were pooled using a random effects model
significant associations were observed between Shannon diversity, number
of observed OTUs, and Shannon evenness and obesity status They were not
observed for the ratio of \emph{Bacteroidetes} and \emph{Firmicutes} or
their individual relative abundances. Although these tests yielded small
P-values, the difference between the Shannon diversity index of
non-obese and obese individuals was 2.07\%. A power analysis
demonstrated that one of the studies had sufficient power to detect a
5\% difference in diversity. When models trained on one dataset were
then tested using the other 9 datasets, the median accuracy varied
between 33.01 and 64.77\% (median=56.67\%). Although there is
statistical support for a relationship between the microbial communities
found in human feces and obesity status, this association is relatively
weak and its detection is confounded by large interpersonal variation
and insufficient sample sizes.

\subsubsection{Importance}\label{importance}

As interest in the human microbiome grows there is an increasing number
of studies that can be used to test numerous hypotheses across human
populations. The hypothesis that variation in the gut microbiota can
explain or be used to predict obesity status has received considerable
attention and is frequently mentioned as an example for the role of the
microbiome in human health. Here we assess this hypothesis using ten
independent studies and find that although there is an association, it
is smaller than can be detected by most microbiome studies. Furthermore,
we directly tested the ability to predict obesity status based on the
composition of an individual's microbiome and find that the median
classification accuracy is between 33.01 and 64.77\%. This type of
analysis can be used to design future studies and expanded to explore
other hypotheses.

\newpage

\subsection{Introduction}\label{introduction}

Obesity is a growing health concern with approximately 20\% of the youth
(aged 2-19) in the United States classified as either overweight or
obese (1). This number increases to approximately 35\% in adults (aged
20 or older) and these statistics have seen little change since 2003
(1). Traditionally the body mass index (BMI) has been used as the
traditional method of classifying individuals as non-obese or obese (2).
Recently, there has been increased interest in the role of the
microbiome in modulating obesity (3, 4). If the microbiome does affect
obesity status, then manipulating the microbiome could have a
significant role in the future treatment of obesity and in helping to
stem the current epidemic.

There have been several studies that report observing a link between the
composition of microbiome and obesity in animal models and in humans.
The first such study used genetically obese mice and observed the ratio
of the relative abundances of Bacteroidetes to Firmicutes (B:F) was
lower in obese mice than lean mice (5). Translation of this result to
humans by the same researchers did not observe this effect, but did find
that obese individuals had a lower diversity than lean individuals (6).
They also showed that the relative abundance of Bacteroidetes and
Firmicutes increased and decreased, respectively, as obese individuals
lost weight while on a fat or carbohydrate restricted diet (7). Two
re-analysis studies interrogated previously published microbiome and
obesity data and concluded that the previously reported differences in
community diversity and B:F among non-obese and obese individuals could
not be generalized (8, 9). Regardless of the results using human
populations, mechanistic studies using animal models that were
manipulated with antibiotics or colonization with varied communities
appears to support the association since direct manipulation of the
communities yielded variation in animal weight (10--13). The purported
association between the differences in the microbiome and obesity have
been widely repeated with little attention given to the lack of a clear
signal in human cohort studies.

The recent publication of additional studies that collected BMI data for
each subject as well as other studies that were not included in the
earlier re-analysis studies offered the opportunity to revisit the
question relating the structure of the human microbiome to obesity
(14--22). One critique of the prior re-analysis studies is that the
authors did not aggregate the results across studies to increase the
effective sample size. It is possible that there were small associations
within each study that were not statistically significant because the
individual studies lacked sufficient power. Alternatively, diversity
metrics may mask the appropriate signal and it is necessary to measure
the association at the level of microbial populations. Walters et al.
(8) demonstrated that Random Forest machine learning models were capable
of predicting obesity status within a single cohort, but did not attempt
to test the models on other cohorts. The purpose of this study was to
perform a meta-analysis of the association between differences in the
microbiome and obesity status by analyzing and applying a more
systematic and synthetic approach than was used previously.

\subsection{Methods}\label{methods}

\textbf{\emph{Literature Review and Study Inclusion.}} We followed the
Preferred Reporting Items for Systematic Reviews and Meta-Analyses
(PRISMA) guidelines to identify studies to include in our meta-analysis
(23). A detailed description of our selection process and the exact
search terms are provided in the Supplement and in figure 1. Briefly, we
searched PubMed for original research studies that involved studying
obesity and the human microbiome. The initial search yielded 187
studies. We identified number\_string{[}n{]} additional studies that
were not designed to explicitly test for an association between the
microbiome and obesity. We then manually curated the 196 studies to
select those studies that included BMI and sequence data. This yielded
10 eligible studies. An additional study was removed from our analysis
because no individuals in the study had a BMI over 30. Among the final
10 studies, 3 were from identified from our PubMed search (6, 15, 20), 5
were originally identified from the 9 studies that did not explicitly
investigate obesity but included BMI data (14, 18, 19, 24, 25), and two
datasets were used (21, 22) because these publications did not
specifically look for any metabolic or obesity conditions but had
control populations and enabled us to help mitigate against publication
biases associated with the bacterial microbiome and obesity. The
number\_string{[}n{]} studies are summarized in Tables 1 and 2.

\textbf{\emph{Sequence Analysis Pipeline.}} All sequence data were
publicly available and were downloaded from the NCBI Sequence Read
Archive, the European Nucleotide Archive, or the investigators' personal
website
(\url{https://gordonlab.wustl.edu/TurnbaughSE/_10/_09/STM/_2009.html}).
In total seven studies used 454 (6, 14, 15, 19, 20, 22, 25) and three
studies used Illumina sequencing (18, 21, 24). All of these studies used
amplification-based 16S rRNA gene sequencing. Among the studies that
sequenced the 16S rRNA gene, the researchers targeted the V1-V2 (19),
V1-V3 (14, 15, 20), V3-V5 (22, 25), V4 {[}(18); (21); {]}, and V3-4 (24)
regions. For those studies where multiple regions were sequenced, we
selected the region that corresponded to the largest number of subjects
(6, 25). We processed the 16S rRNA gene sequence data using a
standardized mothur pipeline. Briefly, our pipelines attempted to follow
previously recommended approaches for 454 and Illumina sequencing data
(26, 27). All sequences were screened for chimeras using UCHIME and
assigned to operational taxonomic units (OTUs) using the average
neighbor algorithm using a 3\% distance threshold (28, 29). All sequence
processing was performed using mothur (v.1.37.0) (30).

\textbf{\emph{Data Analysis.}} We split the overall meta-analysis into
three general strategies using R (3.3.0). First, we followed the
approach employed by Finucane et al (9) and Walters et al (8) where each
study was re-analyzed separately to identify associations between BMI
and the relative abundance of Bacteroidetes and Firmicutes, the ratio of
Bacteroidetes and Firmicutes relative abundances (B:F), Shannon
diversity, observed richness, and Shannon evenness. After each variable
was transformed to fit a normal distribution a two-tailed t-test was
performed for comparison of non-obese and obese individuals (i.e.~BMI
\textgreater{} 35.0). We performed a pooled analysis on these measured
variables using linear random effect models to correct for study effect
to asses differences on the combined dataset between non-obese and obese
groups using the lme4 (v.1.1-12) R package. Next, we compared the
community structure from non-obese and obese individuals using PERMANOVA
analysis of Bray-Curtis distance matrices. This analysis was performed
using the vegan (v.2.3-5) R package. For both analyses, the datasets
were rarefied (N=1000) so that each study within a study had the same
number of sequences. Second, for each study we partitioned the subjects
into a low or high group depending on whether their alpha diversity
metrics were below or above the median value for the study. The relative
risk (RR) was then calculated as the ratio of the number of obese
individuals in the low group to the number of obese individuals in the
high group. We then performed a Fisher exact-test to investigate whether
the RR was significantly different from 1.0 within each study and across
all of the studies using the epiR (0.9-77) and metafor (1.9-8) packages.
Third, we used the AUCRF (1.1) R package to generate Random Forest
models. For each study we developed models using either OTUs or
genus-level phylotypes. The quality of each model was assessed by
measuring the area under the curve (AUC) of the Receive Operating
Characteristic (ROC) using ten-fold cross validation. Because the
genus-level phylotype models were developed using a common reference, it
was possible to use one study's model (i.e.~the training set) to
classify the samples from the other studies (i.e.~the testing sets). The
optimum threshold for the training set was set as the probability
threshold that had the highest combined sensitivity and specificity.
This threshold was then used to calculate the accuracy of the model
applied to the test studies. To generate Receiver Operator
Characteristic (ROC) curves and calculate the accuracy of the models we
used the pROC (1.8) R package. Finally, we performed power and sample
number simulations for different effect sizes for each study using the
pwr (1.1-3) R package and base R functions. We also calculated the
actual sample size needed based on the effect size of each individual
study.

\textbf{\emph{Reproducible methods.}} A detailed and reproducible
description of how the data were processed and analyzed can be found at
\url{https://github.com/SchlossLab/Sze_Obesity_mBio_2016/}.

\subsection{Results}\label{results}

\textbf{\emph{Alpha diversity analysis.}} We calculated the Shannon
diversity index, richness, and Shannon evenness, the relative abundance
of \emph{Bacteroidetes} and \emph{Firmicutes}, and the ratio of their
relative abundance (B:F) for each sample. Once we transformed each of
the six alpha diversity metrics to make them normally distributed, we
used a t-test to identify significant associations between the alpha
diversity metric and whether an individual was obese for each of the ten
studies. The B:F and the relative abundance of \emph{Firmicutes} were
not significantly associated with obesity in any study. We identified 7
P-values less than 0.05: three studies indicated obese individuals had a
lower richness, two studies indicated a significantly lower diversity,
one study indicated a significantly lower evenness, and one study
indicated a significantly higher relative abundance of
\emph{Bacteroidetes} (Figures 2 and S1). These results largely match
those of the Walters and Finucane re-analysis studies. Interestingly,
although only two of the 10 studies observed the previously reported
association between lower diversity and obesity, the other studies
appeared to have the same trend, albeit the differences were not
statistically significant. We used a random effects linear model to
combine the studies using the study as the random effect and found
statistical support for decreased richness, evenness, and diversity
among obese individuals (all P\textless{}0.011). Although there was a
significant relationship between these metrics and diversity and obesity
status, the effect size was quite small. The obese individuals averaged
7.47\% lower richness, 0.88\% lower evenness, and 2.07\% lower
diversity. There were no significant associations when we pooled the
phylum-level metrics across studies. These results indicate that obese
individuals do have a statistically significant lower diversity than
non-obese individuals; however, it is questionable whether the
difference is biologically significant.

\textbf{\emph{Relative risk.}} Building upon the alpha diversity
analysis we calculated the relative risk of being obese based on whether
an individual's alpha diversity metrics were below or above the median
metric for that study. The results using relative risk largely matched
those of using the untransformed alpha diversity data. Across the
number\_string{[}n{]} studies and six metrics, the only significant
relative risk values were the richness, evenness, and diversity values
from the Goodrich study (Figures 3 and S2). Again, although the relative
risk values were not significant for other studies, the values tended to
be above one. When we pooled the data using a random effects model, the
relative risk associated with having a richness, evenness, or diversity
below the median for the population was significantly associated with
obesity (all P\textless{}0.0044). The relative risks associated with
alpha diversity were small. The relative risk of having a low richness
was 1.30 (1.13-1.49), low evenness was 1.20 (1.06-1.37), and low
diversity was 1.27 (1.09-1.48). There were no significant difference in
the phylum-level metrics. Again, the relative risk results indicate that
individuals with a lower richness, evenness, or diversity are at
statistically significant increased risk of being obese, it is
questionable whether that risk is biologically or clinically relevant.

\textbf{\emph{Beta diversity analysis.}} Following the approach used by
the Walters and Finucane re-analysis studies, for each dataset we
calculated a Bray-Curtis distance matrix to measure the difference in
the membership and structure of the individuals from each study. We then
used PERMANOVA to test for a significant differences between the
structure of non-obese and obese individuals. The Escobar, Goodrich, and
Turnbaugh datasets indicated a significant difference in community
structure (all P\textless{}0.05). Because it was not possible to
ascertain the directionality of the difference in community structure
nor perform a pooled analysis using studies that had non-overlapping 16S
rRNA gene sequence regions it is unclear whether these differences
reflect a broader, but perhaps small, shift in community structure
between non-obese and obese individuals.

\textbf{\emph{Development of a microbiome-based classifier of obesity.}}
The Walters re-analysis study suggested that it was possible to classify
individuals as being obese or non-obese based on the composition of
their microbiota. We repeated this analysis with additional datasets
using OTU and genus-level phylotype data. For each study we developed a
Random Forest machine learning model to classify individuals. Using
ten-fold cross validation, the observed AUC values varied between 0.52
and 0.69 indicating a relatively poor ability to classify individuals
(Figure 4A). So that we could test models on other datasets, we trained
models using genus-level phylotype data for each dataset. The the
observed AUC values for the models applied to the training datasets
varied between 0.51 and 0.65, again indicating a relatively poor ability
to classify individuals from the original dataset (Figure 4B). For each
model we identified the probability where the sum of the sensitivity and
specificity was the highest. We then used this to identify a threshold
to calculate the accuracy of the models when applied to the other
number\_string{[}n-1{]} datasets (Figure 5). Although there considerable
variation in accuracy values for each model, the median accuracy for
each model varied between 0.33 (Turnbaugh) and 0.65 (HMP) (median=0.57).
When we considered the number of samples, balance of non-obese and obese
individuals, and region within the 16S rRNA gene it was not possible to
identify factors that predictably affected model performance. The
ability to predict obesity status using the relative abundance of OTUs
and genera in the communities is only marginally better than random.
These results suggest that given the large diversity of microbiome
compositions it is difficult to identify a taxonomic signal that can be
associated with obesity.

\textbf{\emph{Power and Sample Size Estimate Simulations.}} The
inability to detect a difference between non-obese and obese individuals
could be due to the lack of a true effect or because the study had
insufficient statistical power to detect a difference because of
insufficient sampling, large interpersonal variation, and unbalanced
sampling of non-obese and obese individuals. To assess this, we
calculated the power to detect differences of 1, 5, 10, and 15\% in each
of the alpha diversity metrics using the sample sizes used in each of
the studies (Figures 6, S3-S8). Although there is no biological
rationale for these effect sizes, they represent a range that is
plausible. Only the Goodrich study had power greater than 0.80 to detect
a 5\% difference in Shannon diversity and six of the studies had enough
power to detect a 10\% difference (Figure 6). None of the studies had
sufficient power to detect a 15\% difference between B:F values (Figure
S5). In fact, the maximum power among any of the studies to detect a
15\% difference in B:F values was 0.25. Among the tests for relative
risk, none of the studies had sufficient power to detect a Cohen's d of
0.10 and only two studies had sufficient power to detect a Cohen's d of
0.15. We next estimated how many individuals would need to have been
sampled to have sufficient power to detect the four effect sizes
assuming the observed interpersonal variation from each study and
balanced sampling between the two groups. To detect a 1, 5, 10, and 15\%
difference in Shannon index, the median sampling effort per group was
approximately 3,400, 140, 35, and 16 individuals, respectively. To
detect a 1, 5, 10, and 15\% difference in B:F values, the median
sampling effort per group was approximately 160,000, 6,300, 1,600, and
700 individuals, respectively. To detect a 1, 5, 10, and 15\% difference
in relative risk values using Shannon diversity, the median sampling
effort per group was approximately 39,000, 1,500, 380, and 170
individuals, respectively. These estimates indicate that most microbiome
studies are underpowered to detect modest effect sizes using either
metric. In the case of obesity, the studies were underpowered to detect
the \texttt{signif(range\_effect\_size,\ 1){[}1{]}} to 6\% difference in
diversity that was observed across the studies.

\subsection{Discussion}\label{discussion}

Our meta-analysis helps to provide a clarity to the ongoing debate of
whether or not there are specific microbiome-based markers that can be
associated with obesity. We performed an extensive literature review of
the existing studies on the microbiome and obesity and perform a
meta-analysis on the studies that remained based on our inclusion and
exclusion criteria. By statistically pooling the data from ten studies,
we observed significant, but small, relationships between richness,
evenness, and diversity and obesity status as well as the relative risk
of being obese based on these metrics. We also generated Random Forest
machine learning models trained on each dataset and tested on the
remaining datasets. This analysis demonstrated that the ability to
reliably classify individuals as being obese based on the composition of
their microbiome is limited. Finally, we assessed the ability of each
study to detect defined differences in alpha diversity and observed that
most studies were underpowered to detect modest effect sizes.
Considering these datasets are among the largest published, it appears
that most microbiome studies are underpowered to detect differences in
alpha diversity.

Alpha diversity metrics are attractive because they distill a complex
dataset to a single value. For example, diversity is a measure of the
entropy in a community and integrates richness and evenness information.
Two communities with little taxonomic similarity can have the same
diversity. Among ecologists the relevance of these metrics is questioned
because it is difficult to ascribe a mechanistic interpretation to their
relationship with stability or disease. Regardless, the concept of a
biologically significant effect size needs to be developed among
microbiome researchers. Alternative metrics could include the ability to
detect a defined difference in the relative abundance of an OTU
representing a defined relative abundance. What makes for a biologically
significant difference or relative abundance is an important point that
has yet to be discussed in the microbiome field. The use of
operationally defined effect sizes should be adequate until it is
possible to decide upon an accepted practice.

By selecting a range of possible effect sizes, we were able to
demonstrate that most studies are underpowered to detect modest
differences in alpha diversity metrics and phylum-level relative
abundances. Several factors interact to limit the power of microbiome
studies. There is wide interpersonal variation in the diversity and
structure of the human microbiome. In addition, the common experimental
designs limit their power. As we observed, most of the studies included
in our analysis were unbalanced for the variable that we were interested
in. This was also true of those studies that originally sought to
identify associations with obesity. Even with a balanced design, we
showed that it was necessary to obtain approximately 140 and 6,300
sequences per sample to detect a 5\% difference in Shannon diversity or
B:F, respectively. It was interesting that these sample sizes agreed
across studies regardless of their sequencing method, region with in the
16S rRNA gene, or subject population (Figure 6). This suggests that
regardless of the treatment or category, these sample sizes represent a
good starting point for subject recruitment when using stool samples.
Unfortunately, few studies have been published with this level of
subject recruitment. This is troubling since the positive predictive
rate of a significant finding in an underpowered study is small leading
to results that cannot be reproduced (31). Future microbiome studies
should articulate the basis for their experimental design.

Two previous reviews (8, 9) have stated that there was not a consistent
association between alpha diversity and obesity; however, neither of
these studies made an attempt to pool the existing data together to try
and harness the additional power that this would give and they did not
assess whether the studies were sufficiently powered to detect a
difference. Our analysis also used 16S rRNA gene sequence data from ten
studies whereas the Finucane study used 16S rRNA gene sequence data from
3 studies (7, 10, 25) and a metagenomic study (32) and the Walters study
used 16S rRNA gene sequence data from 5 studies (10, 15, 19, 25, 33);
two studies were included in both analyses (10, 25). Our analysis
included 4 of these studies (10, 15, 19, 25) and excluded 3 of the
studies because they were too small (7), only utilized metagenomic data
(32), or used short single read Illumina HiSeq data that has a high
error rate making it untractable for \emph{de novo} OTU clustering (33).
The additional seven datasets were published after the two reviews were
performed and include datasets with more samples than were found in the
original studies. Our collection of ten studies allowed us to largely
use the same sequence analysis pipeline for all datasets and relied
heavily on the availability of public data and access to metadata that
included variables beyond the needs of the original study. To execute
this analysis, we created an automated data analysis pipeline, which can
be easily updated to add additional studies as they become available.
Similarly, it would be possible to adapt this pipeline to other body
sites and treatment or variables (e.g.~subject's sex or age).

Similar to our study, the Walters analysis (8), the authors generated
Random Forest machine learning models to differentiate between non-obese
and obese individuals. They obtained similar AUC values to our analysis;
however, they did not attempt to test these models on the other studies
in their analysis. When we performed the inter-dataset cross validation
the median accuracy across datasets was only 56.67\% indicating that the
models did a poor job when applied to other datasets. This could be due
to differences in subject populations and methods. Considering the
median AUC for models trained and tested on the same data with ten-fold
cross validation only varied between 0.51 and 0.65 and there was not a
strong signal in the alpha diversity data, we suspect that there is
insufficient signal to reliably classify individuals.

Although we failed to find an effect it is not realistic to necessarily
state that there is no microbiome impact on obesity. There is strong
evidence in murine models of obesity that the microbiome and level of
adiposity can be manipulated via genetic manipulation of the animal and
manipulation of the community through antibiotics or colonizing germ
free mice with diverse fecal material from human donors (5, 10--13).
These studies appear to conflict with the observations using human
subjects. Recalling the large interpersonal variation in the structure
of the microbiome, it is possible that each individual has their own
signatures of obesity. Alternatively, it could be that the involvement
of the microbiome in obesity is at the level of a common set of
metabolites that can be produced from different structures of the
microbiome.

\subsection{Acknowledgements}\label{acknowledgements}

The authors would like to thank Nielson Baxter and Shawn Whitefield for
their suggestions on the development of the manuscript. We are grateful
to the authors of the studies used in our meta-analysis who have made
their data publicly available or available to us directly. Without their
forethought studies such as this would not be possible. This work was
supported in part by funding from the National Institutes of Health to
PDS (U01AI2425501 and P30DK034933).

\newpage

\textbf{Table 1. Summary Demographics of Individuals used in the
Meta-analysis.}

\newpage

\textbf{Figure 1: PRISMA flow diagram of total records searched (34).}

\textbf{Figure 2: Individual and combined comparison of obese and
non-obese groups for Shannon diversity (A) and B:F (B).}

\textbf{Figure 3: Meta analysis of the relative risk of obesity based on
Shannon diversity (A) or B:F (B).}

\textbf{Figure 4: ROC curves for each study based on classification of
non-obese or obese groups using OTUs (A) or genus-level classification
(B).}

\textbf{Figure 5: Overall accuracy of each study to predict non-obese
and obese individuals based on that study's Random Forest machine
learning model applied to each of the other studies.}

\textbf{Figure 6: Power (A) and sample size simulations (B) for Shannon
diversity for differentiating between non-obese versus obese for effect
sizes of 1, 5, 10, and 15\%.} Power calculations use the sampling
distribution from the original studies and the sample size estimations
assume an equal amount of sampling from each treatment group.

\newpage

\textbf{Figure S1: Individual and Combined comparison of Obese and
Non-Obese groups Based on Evenness (A), Richness (B), or the Relative
Abundance of Bacteroidetes (C) and Firmictues (D).}

\textbf{Figure S2: Meta Analysis of the Relative Risk of Obesity Based
on Evenness (A), Richness (B), or the Relative Abundance of
Bacteroidetes (C) and Firmictues (D).}

\textbf{Figure S3: Power (A) and sample size simulations (B) for B:F for
differentiating between non-obese versus obese for effect sizes of 1, 5,
10, and 15\%.} Power calculations use the sampling distribution from the
original studies and the sample size estimations assume an equal amount
of sampling from each treatment group.

\textbf{Figure S4: Power (A) and sample size simulations (B) for
richness for differentiating between non-obese versus obese for effect
sizes of 1, 5, 10, and 15\%.} Power calculations use the sampling
distribution from the original studies and the sample size estimations
assume an equal amount of sampling from each treatment group.

\textbf{Figure S5: Power (A) and sample size simulations (B) for
evenness for differentiating between non-obese versus obese for effect
sizes of 1, 5, 10, and 15\%.} Power calculations use the sampling
distribution from the original studies and the sample size estimations
assume an equal amount of sampling from each treatment group.

\textbf{Figure S6: Power (A) and sample size simulations (B) for the
relative abundance of Bacteroidetes for differentiating between
non-obese versus obese for effect sizes of 1, 5, 10, and 15\%.} Power
calculations use the sampling distribution from the original studies and
the sample size estimations assume an equal amount of sampling from each
treatment group.

\textbf{Figure S7: Power (A) and sample size simulations (B) for the
relative abundance of Firmicutes for differentiating between non-obese
versus obese for effect sizes of 1, 5, 10, and 15\%.} Power calculations
use the sampling distribution from the original studies and the sample
size estimations assume an equal amount of sampling from each treatment
group.

\textbf{Figure S8: Power (A) and sample size simulations (B) for
relative risk of obesity based on Shannon diversity.} Power calculations
use the sampling distribution from the original studies and the sample
size estimations assume an equal amount of sampling from each treatment
group.

\newpage

\hyperdef{}{references}{\label{references}}
\subsection*{References}\label{references}
\addcontentsline{toc}{subsection}{References}

\hyperdef{}{ref-ogdenux5fprevalenceux5f2014}{\label{ref-ogdenux5fprevalenceux5f2014}}
1. \textbf{Ogden CL}, \textbf{Carroll MD}, \textbf{Kit BK},
\textbf{Flegal KM}. 2014. Prevalence of childhood and adult obesity in
the United States, 2011-2012. JAMA \textbf{311}:806--814.
doi:\url{http://doi.org/10.1001/jama.2014.732}.

\hyperdef{}{ref-lichtashux5fbodyux5f2013}{\label{ref-lichtashux5fbodyux5f2013}}
2. \textbf{Lichtash CT}, \textbf{Cui J}, \textbf{Guo X}, \textbf{Chen
Y-DI}, \textbf{Hsueh WA}, \textbf{Rotter JI}, \textbf{Goodarzi MO}.
2013. Body adiposity index versus body mass index and other
anthropometric traits as correlates of cardiometabolic risk factors.
PloS One \textbf{8}:e65954.
doi:\url{http://doi.org/10.1371/journal.pone.0065954}.

\hyperdef{}{ref-braheux5fcanux5f2016}{\label{ref-braheux5fcanux5f2016}}
3. \textbf{Brahe LK}, \textbf{Astrup A}, \textbf{Larsen LH}. 2016. Can
We Prevent Obesity-Related Metabolic Diseases by Dietary Modulation of
the Gut Microbiota? Advances in Nutrition (Bethesda, Md)
\textbf{7}:90--101. doi:\url{http://doi.org/10.3945/an.115.010587}.

\hyperdef{}{ref-drorux5fmicrobiotaux5f2016}{\label{ref-drorux5fmicrobiotaux5f2016}}
4. \textbf{Dror T}, \textbf{Dickstein Y}, \textbf{Dubourg G},
\textbf{Paul M}. 2016. Microbiota manipulation for weight change.
Microbial Pathogenesis.
doi:\url{http://doi.org/10.1016/j.micpath.2016.01.002}.

\hyperdef{}{ref-leyux5fobesityux5f2005}{\label{ref-leyux5fobesityux5f2005}}
5. \textbf{Ley RE}, \textbf{Bäckhed F}, \textbf{Turnbaugh P},
\textbf{Lozupone CA}, \textbf{Knight RD}, \textbf{Gordon JI}. 2005.
Obesity alters gut microbial ecology. Proceedings of the National
Academy of Sciences of the United States of America
\textbf{102}:11070--11075.
doi:\url{http://doi.org/10.1073/pnas.0504978102}.

\hyperdef{}{ref-turnbaughux5fcoreux5f2009}{\label{ref-turnbaughux5fcoreux5f2009}}
6. \textbf{Turnbaugh PJ}, \textbf{Hamady M}, \textbf{Yatsunenko T},
\textbf{Cantarel BL}, \textbf{Duncan A}, \textbf{Ley RE}, \textbf{Sogin
ML}, \textbf{Jones WJ}, \textbf{Roe BA}, \textbf{Affourtit JP},
\textbf{Egholm M}, \textbf{Henrissat B}, \textbf{Heath AC},
\textbf{Knight R}, \textbf{Gordon JI}. 2009. A core gut microbiome in
obese and lean twins. Nature \textbf{457}:480--484.
doi:\url{http://doi.org/10.1038/nature07540}.

\hyperdef{}{ref-leyux5fmicrobialux5f2006}{\label{ref-leyux5fmicrobialux5f2006}}
7. \textbf{Ley RE}, \textbf{Turnbaugh PJ}, \textbf{Klein S},
\textbf{Gordon JI}. 2006. Microbial ecology: Human gut microbes
associated with obesity. Nature \textbf{444}:1022--1023.
doi:\url{http://doi.org/10.1038/4441022a}.

\hyperdef{}{ref-waltersux5fmeta-analysesux5f2014}{\label{ref-waltersux5fmeta-analysesux5f2014}}
8. \textbf{Walters WA}, \textbf{Xu Z}, \textbf{Knight R}. 2014.
Meta-analyses of human gut microbes associated with obesity and IBD.
FEBS letters \textbf{588}:4223--4233.
doi:\url{http://doi.org/10.1016/j.febslet.2014.09.039}.

\hyperdef{}{ref-finucaneux5ftaxonomicux5f2014}{\label{ref-finucaneux5ftaxonomicux5f2014}}
9. \textbf{Finucane MM}, \textbf{Sharpton TJ}, \textbf{Laurent TJ},
\textbf{Pollard KS}. 2014. A taxonomic signature of obesity in the
microbiome? Getting to the guts of the matter. PloS One
\textbf{9}:e84689.
doi:\url{http://doi.org/10.1371/journal.pone.0084689}.

\hyperdef{}{ref-turnbaughux5fobesity-associatedux5f2006}{\label{ref-turnbaughux5fobesity-associatedux5f2006}}
10. \textbf{Turnbaugh PJ}, \textbf{Ley RE}, \textbf{Mahowald MA},
\textbf{Magrini V}, \textbf{Mardis ER}, \textbf{Gordon JI}. 2006. An
obesity-associated gut microbiome with increased capacity for energy
harvest. Nature \textbf{444}:1027--31.
doi:\url{http://doi.org/10.1038/nature05414}.

\hyperdef{}{ref-Koren2012}{\label{ref-Koren2012}}
11. \textbf{Koren O}, \textbf{Goodrich JK}, \textbf{Cullender TC},
\textbf{Spor A}, \textbf{Laitinen K}, \textbf{Bäckhed HK},
\textbf{Gonzalez A}, \textbf{Werner JJ}, \textbf{Angenent LT},
\textbf{Knight R}, \textbf{Bäckhed F}, \textbf{Isolauri E},
\textbf{Salminen S}, \textbf{Ley RE}. 2012. Host remodeling of the gut
microbiome and metabolic changes during pregnancy. Cell
\textbf{150}:470--480.
doi:\url{http://doi.org/10.1016/j.cell.2012.07.008}.

\hyperdef{}{ref-Cox2014}{\label{ref-Cox2014}}
12. \textbf{Cox LM}, \textbf{Yamanishi S}, \textbf{Sohn J},
\textbf{Alekseyenko AV}, \textbf{Leung JM}, \textbf{Cho I}, \textbf{Kim
SG}, \textbf{Li H}, \textbf{Gao Z}, \textbf{Mahana D}, \textbf{Rodriguez
JGZ}, \textbf{Rogers AB}, \textbf{Robine N}, \textbf{Loke P},
\textbf{Blaser MJ}. 2014. Altering the intestinal microbiota during a
critical developmental window has lasting metabolic consequences. Cell
\textbf{158}:705--721.
doi:\url{http://doi.org/10.1016/j.cell.2014.05.052}.

\hyperdef{}{ref-Mahana2016}{\label{ref-Mahana2016}}
13. \textbf{Mahana D}, \textbf{Trent CM}, \textbf{Kurtz ZD},
\textbf{Bokulich NA}, \textbf{Battaglia T}, \textbf{Chung J},
\textbf{Müller CL}, \textbf{Li H}, \textbf{Bonneau RA}, \textbf{Blaser
MJ}. 2016. Antibiotic perturbation of the murine gut microbiome enhances
the adiposity, insulin resistance, and liver disease associated with
high-fat diet. Genome Medicine \textbf{8}.
doi:\url{http://doi.org/10.1186/s13073-016-0297-9}.

\hyperdef{}{ref-rossux5f16sux5f2015}{\label{ref-rossux5f16sux5f2015}}
14. \textbf{Ross MC}, \textbf{Muzny DM}, \textbf{McCormick JB},
\textbf{Gibbs RA}, \textbf{Fisher-Hoch SP}, \textbf{Petrosino JF}. 2015.
16S gut community of the Cameron County Hispanic Cohort. Microbiome
\textbf{3}:7. doi:\url{http://doi.org/10.1186/s40168-015-0072-y}.

\hyperdef{}{ref-zupancicux5fanalysisux5f2012}{\label{ref-zupancicux5fanalysisux5f2012}}
15. \textbf{Zupancic ML}, \textbf{Cantarel BL}, \textbf{Liu Z},
\textbf{Drabek EF}, \textbf{Ryan KA}, \textbf{Cirimotich S},
\textbf{Jones C}, \textbf{Knight R}, \textbf{Walters WA},
\textbf{Knights D}, \textbf{Mongodin EF}, \textbf{Horenstein RB},
\textbf{Mitchell BD}, \textbf{Steinle N}, \textbf{Snitker S},
\textbf{Shuldiner AR}, \textbf{Fraser CM}. 2012. Analysis of the gut
microbiota in the old order Amish and its relation to the metabolic
syndrome. PloS One \textbf{7}:e43052.
doi:\url{http://doi.org/10.1371/journal.pone.0043052}.

\hyperdef{}{ref-namux5fcomparativeux5f2011}{\label{ref-namux5fcomparativeux5f2011}}
16. \textbf{Nam Y-D}, \textbf{Jung M-J}, \textbf{Roh SW}, \textbf{Kim
M-S}, \textbf{Bae J-W}. 2011. Comparative analysis of Korean human gut
microbiota by barcoded pyrosequencing. PloS One \textbf{6}:e22109.
doi:\url{http://doi.org/10.1371/journal.pone.0022109}.

\hyperdef{}{ref-arumugamux5fenterotypesux5f2011}{\label{ref-arumugamux5fenterotypesux5f2011}}
17. \textbf{Arumugam M}, \textbf{Raes J}, \textbf{Pelletier E},
\textbf{Le Paslier D}, \textbf{Yamada T}, \textbf{Mende DR},
\textbf{Fernandes GR}, \textbf{Tap J}, \textbf{Bruls T}, \textbf{Batto
J-M}, \textbf{Bertalan M}, \textbf{Borruel N}, \textbf{Casellas F},
\textbf{Fernandez L}, \textbf{Gautier L}, \textbf{Hansen T},
\textbf{Hattori M}, \textbf{Hayashi T}, \textbf{Kleerebezem M},
\textbf{Kurokawa K}, \textbf{Leclerc M}, \textbf{Levenez F},
\textbf{Manichanh C}, \textbf{Nielsen HB}, \textbf{Nielsen T},
\textbf{Pons N}, \textbf{Poulain J}, \textbf{Qin J},
\textbf{Sicheritz-Ponten T}, \textbf{Tims S}, \textbf{Torrents D},
\textbf{Ugarte E}, \textbf{Zoetendal EG}, \textbf{Wang J},
\textbf{Guarner F}, \textbf{Pedersen O}, \textbf{Vos WM de},
\textbf{Brunak S}, \textbf{Doré J}, \textbf{MetaHIT Consortium},
\textbf{Antolín M}, \textbf{Artiguenave F}, \textbf{Blottiere HM},
\textbf{Almeida M}, \textbf{Brechot C}, \textbf{Cara C},
\textbf{Chervaux C}, \textbf{Cultrone A}, \textbf{Delorme C},
\textbf{Denariaz G}, \textbf{Dervyn R}, \textbf{Foerstner KU},
\textbf{Friss C}, \textbf{Guchte M van de}, \textbf{Guedon E},
\textbf{Haimet F}, \textbf{Huber W}, \textbf{Hylckama-Vlieg J van},
\textbf{Jamet A}, \textbf{Juste C}, \textbf{Kaci G}, \textbf{Knol J},
\textbf{Lakhdari O}, \textbf{Layec S}, \textbf{Le Roux K},
\textbf{Maguin E}, \textbf{Mérieux A}, \textbf{Melo Minardi R},
\textbf{M'rini C}, \textbf{Muller J}, \textbf{Oozeer R},
\textbf{Parkhill J}, \textbf{Renault P}, \textbf{Rescigno M},
\textbf{Sanchez N}, \textbf{Sunagawa S}, \textbf{Torrejon A},
\textbf{Turner K}, \textbf{Vandemeulebrouck G}, \textbf{Varela E},
\textbf{Winogradsky Y}, \textbf{Zeller G}, \textbf{Weissenbach J},
\textbf{Ehrlich SD}, \textbf{Bork P}. 2011. Enterotypes of the human gut
microbiome. Nature \textbf{473}:174--180.
doi:\url{http://doi.org/10.1038/nature09944}.

\hyperdef{}{ref-goodrichux5fhumanux5f2014}{\label{ref-goodrichux5fhumanux5f2014}}
18. \textbf{Goodrich JK}, \textbf{Waters JL}, \textbf{Poole AC},
\textbf{Sutter JL}, \textbf{Koren O}, \textbf{Blekhman R},
\textbf{Beaumont M}, \textbf{Van Treuren W}, \textbf{Knight R},
\textbf{Bell JT}, \textbf{Spector TD}, \textbf{Clark AG}, \textbf{Ley
RE}. 2014. Human genetics shape the gut microbiome. Cell
\textbf{159}:789--799.
doi:\url{http://doi.org/10.1016/j.cell.2014.09.053}.

\hyperdef{}{ref-wuux5flinkingux5f2011}{\label{ref-wuux5flinkingux5f2011}}
19. \textbf{Wu GD}, \textbf{Chen J}, \textbf{Hoffmann C},
\textbf{Bittinger K}, \textbf{Chen Y-Y}, \textbf{Keilbaugh SA},
\textbf{Bewtra M}, \textbf{Knights D}, \textbf{Walters WA},
\textbf{Knight R}, \textbf{Sinha R}, \textbf{Gilroy E}, \textbf{Gupta
K}, \textbf{Baldassano R}, \textbf{Nessel L}, \textbf{Li H},
\textbf{Bushman FD}, \textbf{Lewis JD}. 2011. Linking long-term dietary
patterns with gut microbial enterotypes. Science (New York, NY)
\textbf{334}:105--108. doi:\url{http://doi.org/10.1126/science.1208344}.

\hyperdef{}{ref-escobarux5fgutux5f2014}{\label{ref-escobarux5fgutux5f2014}}
20. \textbf{Escobar JS}, \textbf{Klotz B}, \textbf{Valdes BE},
\textbf{Agudelo GM}. 2014. The gut microbiota of Colombians differs from
that of Americans, Europeans and Asians. BMC microbiology
\textbf{14}:311. doi:\url{http://doi.org/10.1186/s12866-014-0311-6}.

\hyperdef{}{ref-baxterux5fmicrobiota-basedux5f2016}{\label{ref-baxterux5fmicrobiota-basedux5f2016}}
21. \textbf{Baxter NT}, \textbf{Ruffin MT}, \textbf{Rogers MAM},
\textbf{Schloss PD}. 2016. Microbiota-based model improves the
sensitivity of fecal immunochemical test for detecting colonic lesions.
Genome Medicine \textbf{8}:37.
doi:\url{http://doi.org/10.1186/s13073-016-0290-3}.

\hyperdef{}{ref-schubertux5fmicrobiomeux5f2014}{\label{ref-schubertux5fmicrobiomeux5f2014}}
22. \textbf{Schubert AM}, \textbf{Rogers MAM}, \textbf{Ring C},
\textbf{Mogle J}, \textbf{Petrosino JP}, \textbf{Young VB},
\textbf{Aronoff DM}, \textbf{Schloss PD}. 2014. Microbiome data
distinguish patients with Clostridium difficile infection and non-C.
difficile-associated diarrhea from healthy controls. mBio
\textbf{5}:e01021--01014.
doi:\url{http://doi.org/10.1128/mBio.01021-14}.

\hyperdef{}{ref-moherux5fpreferredux5f2010}{\label{ref-moherux5fpreferredux5f2010}}
23. \textbf{Moher D}, \textbf{Liberati A}, \textbf{Tetzlaff J},
\textbf{Altman DG}, \textbf{PRISMA Group}. 2010. Preferred reporting
items for systematic reviews and meta-analyses: The PRISMA statement.
International Journal of Surgery (London, England) \textbf{8}:336--341.
doi:\url{http://doi.org/10.1016/j.ijsu.2010.02.007}.

\hyperdef{}{ref-zeeviux5fpersonalizedux5f2015}{\label{ref-zeeviux5fpersonalizedux5f2015}}
24. \textbf{Zeevi D}, \textbf{Korem T}, \textbf{Zmora N},
\textbf{Israeli D}, \textbf{Rothschild D}, \textbf{Weinberger A},
\textbf{Ben-Yacov O}, \textbf{Lador D}, \textbf{Avnit-Sagi T},
\textbf{Lotan-Pompan M}, \textbf{Suez J}, \textbf{Mahdi JA},
\textbf{Matot E}, \textbf{Malka G}, \textbf{Kosower N}, \textbf{Rein M},
\textbf{Zilberman-Schapira G}, \textbf{Dohnalová L},
\textbf{Pevsner-Fischer M}, \textbf{Bikovsky R}, \textbf{Halpern Z},
\textbf{Elinav E}, \textbf{Segal E}. 2015. Personalized Nutrition by
Prediction of Glycemic Responses. Cell \textbf{163}:1079--1094.
doi:\url{http://doi.org/10.1016/j.cell.2015.11.001}.

\hyperdef{}{ref-humanux5fmicrobiomeux5fprojectux5fconsortiumux5fstructureux5f2012}{\label{ref-humanux5fmicrobiomeux5fprojectux5fconsortiumux5fstructureux5f2012}}
25. \textbf{Human Microbiome Project Consortium}. 2012. Structure,
function and diversity of the healthy human microbiome. Nature
\textbf{486}:207--214. doi:\url{http://doi.org/10.1038/nature11234}.

\hyperdef{}{ref-Kozichux5f2013}{\label{ref-Kozichux5f2013}}
26. \textbf{Kozich JJ}, \textbf{Westcott SL}, \textbf{Baxter NT},
\textbf{Highlander SK}, \textbf{Schloss PD}. 2013. Development of a
dual-index sequencing strategy and curation pipeline for analyzing
amplicon sequence data on the MiSeq Illumina sequencing platform.
Applied and environmental microbiology \textbf{79}:5112--5120.

\hyperdef{}{ref-Schloss2011}{\label{ref-Schloss2011}}
27. \textbf{Schloss PD}, \textbf{Gevers D}, \textbf{Westcott SL}. 2011.
Reducing the effects of PCR amplification and sequencing artifacts on
16S rRNA-based studies. PLoS ONE \textbf{6}:e27310.
doi:\url{http://doi.org/10.1371/journal.pone.0027310}.

\hyperdef{}{ref-Edgar2011}{\label{ref-Edgar2011}}
28. \textbf{Edgar RC}, \textbf{Haas BJ}, \textbf{Clemente JC},
\textbf{Quince C}, \textbf{Knight R}. 2011. UCHIME improves sensitivity
and speed of chimera detection. Bioinformatics \textbf{27}:2194--2200.
doi:\url{http://doi.org/10.1093/bioinformatics/btr381}.

\hyperdef{}{ref-Westcott2015}{\label{ref-Westcott2015}}
29. \textbf{Westcott SL}, \textbf{Schloss PD}. 2015. De novo clustering
methods outperform reference-based methods for assigning 16S rRNA gene
sequences to operational taxonomic units. PeerJ \textbf{3}:e1487.
doi:\url{http://doi.org/10.7717/peerj.1487}.

\hyperdef{}{ref-mothur}{\label{ref-mothur}}
30. \textbf{Schloss PD}, \textbf{Westcott SL}, \textbf{Ryabin T},
\textbf{Hall JR}, \textbf{Hartmann M}, \textbf{Hollister EB},
\textbf{Lesniewski RA}, \textbf{Oakley BB}, \textbf{Parks DH},
\textbf{Robinson CJ}, \textbf{others}. 2009. Introducing mothur:
open-source, platform-independent, community-supported software for
describing and comparing microbial communities. Applied and
environmental microbiology \textbf{75}:7537--7541.

\hyperdef{}{ref-Ioannidis2005}{\label{ref-Ioannidis2005}}
31. \textbf{Ioannidis JPA}. 2005. Why most published research findings
are false. PLoS Med \textbf{2}:e124.
doi:\url{http://doi.org/10.1371/journal.pmed.0020124}.

\hyperdef{}{ref-Qin2010}{\label{ref-Qin2010}}
32. \textbf{Qin J}, \textbf{Li R}, \textbf{Raes J}, \textbf{Arumugam M},
\textbf{Burgdorf KS}, \textbf{Manichanh C}, \textbf{Nielsen T},
\textbf{Pons N}, \textbf{Levenez F}, \textbf{Yamada T}, \textbf{Mende
DR}, \textbf{Li J}, \textbf{Xu J}, \textbf{Li S}, \textbf{Li D},
\textbf{Cao J}, \textbf{Wang B}, \textbf{Liang H}, \textbf{Zheng H},
\textbf{Xie Y}, \textbf{Tap J}, \textbf{Lepage P}, \textbf{Bertalan M},
\textbf{Batto J-M}, \textbf{Hansen T}, \textbf{Paslier DL},
\textbf{Linneberg A}, \textbf{Nielsen HB}, \textbf{Pelletier E},
\textbf{Renault P}, \textbf{Sicheritz-Ponten T}, \textbf{Turner K},
\textbf{Zhu H}, \textbf{Yu C}, \textbf{Li S}, \textbf{Jian M},
\textbf{Zhou Y}, \textbf{Li Y}, \textbf{Zhang X}, \textbf{Li S},
\textbf{Qin N}, \textbf{Yang H}, \textbf{Wang J}, \textbf{Brunak S},
\textbf{Doré J}, \textbf{Guarner F}, \textbf{Kristiansen K},
\textbf{Pedersen O}, \textbf{Parkhill J}, \textbf{Weissenbach J},
\textbf{Antolin M}, \textbf{Artiguenave F}, \textbf{Blottiere H},
\textbf{Borruel N}, \textbf{Bruls T}, \textbf{Casellas F},
\textbf{Chervaux C}, \textbf{Cultrone A}, \textbf{Delorme C},
\textbf{Denariaz G}, \textbf{Dervyn R}, \textbf{Forte M}, \textbf{Friss
C}, \textbf{Guchte M van de}, \textbf{Guedon E}, \textbf{Haimet F},
\textbf{Jamet A}, \textbf{Juste C}, \textbf{Kaci G}, \textbf{Kleerebezem
M}, \textbf{Knol J}, \textbf{Kristensen M}, \textbf{Layec S},
\textbf{Roux KL}, \textbf{Leclerc M}, \textbf{Maguin E}, \textbf{Minardi
RM}, \textbf{Oozeer R}, \textbf{Rescigno M}, \textbf{Sanchez N},
\textbf{Tims S}, \textbf{Torrejon T}, \textbf{Varela E}, \textbf{Vos W
de}, \textbf{Winogradsky Y}, \textbf{Zoetendal E}, \textbf{Bork P},
\textbf{Ehrlich SD}, \textbf{Wang J}. 2010. A human gut microbial gene
catalogue established by metagenomic sequencing. Nature
\textbf{464}:59--65. doi:\url{http://doi.org/10.1038/nature08821}.

\hyperdef{}{ref-yatsunenkoux5fhumanux5f2012}{\label{ref-yatsunenkoux5fhumanux5f2012}}
33. \textbf{Yatsunenko T}, \textbf{Rey FE}, \textbf{Manary MJ},
\textbf{Trehan I}, \textbf{Dominguez-Bello MG}, \textbf{Contreras M},
\textbf{Magris M}, \textbf{Hidalgo G}, \textbf{Baldassano RN},
\textbf{Anokhin AP}, \textbf{Heath AC}, \textbf{Warner B},
\textbf{Reeder J}, \textbf{Kuczynski J}, \textbf{Caporaso JG},
\textbf{Lozupone CA}, \textbf{Lauber C}, \textbf{Clemente JC},
\textbf{Knights D}, \textbf{Knight R}, \textbf{Gordon JI}. 2012. Human
gut microbiome viewed across age and geography. Nature
\textbf{486}:222--227. doi:\url{http://doi.org/10.1038/nature11053}.

\hyperdef{}{ref-Moher2009}{\label{ref-Moher2009}}
34. \textbf{Moher D}, \textbf{Liberati A}, \textbf{Tetzlaff J},
\textbf{Altman DG}. 2009. Preferred reporting items for systematic
reviews and meta-analyses: The PRISMA statement. PLoS Med
\textbf{6}:e1000097.
doi:\url{http://doi.org/10.1371/journal.pmed.1000097}.

\end{document}
